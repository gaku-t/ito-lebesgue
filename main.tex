\documentclass{amsart}

\usepackage{amsmath, amssymb}
\usepackage{framed}

\theoremstyle{definition}
\newtheorem{ans}{問}
\numberwithin{ans}{section}

\newtheorem{mynote}{行間メモ}
\numberwithin{mynote}{section}
\newenvironment{note}
  {\begin{leftbar}\begin{mynote}}
  {\end{mynote}\end{leftbar}}

\newcommand{\fakesection}[1]{%
  \par\refstepcounter{section}% Increase section counter
  \sectionmark{#1}% Add section mark (header)
  \addcontentsline{toc}{section}{\protect\numberline{\thesection}#1}% Add section to ToC
  % Add more content here, if needed.
}


\begin{document}

\fakesection{§1 Lebesgue (ルベーグ) 測度とは何か}


\fakesection{§2 空間とその部分集合}

\begin{ans}
  「$x \in A \cup B$」
  $\Leftrightarrow$
  「$x \in A$ または $x \in B$」
  $\Leftrightarrow$
  「$x \in A$ または $x \in B \cap A^c$」
  $\Leftrightarrow$
  「$x \in A + (B \cap A^c)$」
\end{ans}

\end{document}
